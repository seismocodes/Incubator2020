% !TEX encoding = UTF-8 Unicode
\documentclass{beamer}

\usepackage{amsmath}
\usepackage{color}
\usepackage{gensymb}
\usepackage{hyperref}
\usepackage{textcomp}
\usepackage{wasysym}

\usepackage{listings}
\lstset{language=Python}

\usetheme{Warsaw}

\newcommand{\btVFill}{\vskip0pt plus 1filll}

\title[Daily monitoring of low-frequency earthquake activity]{Daily monitoring of low-frequency earthquake activity}
\author{Ariane Ducellier, Scott Henderson}
\date{Winter 2020 Incubator project}

\begin{document}

	\begin{frame}
		\titlepage
	\end{frame}

	\begin{frame}
		\frametitle{Low-frequency earthquakes (LFEs)}
		\begin{itemize}
			\item Small magnitude (M $\sim$ 1)
			\item Dominant frequency low (1-10 Hz) compared with that of ordinary tiny earthquakes (up to 20 Hz)
			\item Source located on the plate boundary
			\item Grouped into families of events, with all the earthquakes of a given family originating from the same small patch on the plate interface
			\item Recurrence more or less episodic in a bursty manner
		\end{itemize}
	\end{frame}

	\begin{frame}
		\frametitle{What we do now}
		\begin{itemize}
			\item Download seismic data for a given period of time
			\item Analyze data and find time of occurrence of LFEs
			\item Create a catalog of LFEs for a given period of time and publish it
		\end{itemize}
	\end{frame}

	\begin{frame}
		\frametitle{What we aim to do}
		\begin{itemize}
			\item Low-frequency earthquake occur and are recorded by permanent seismic stations every day
			\item On a daily basis:
			\begin{itemize}
				\item Download seismic data from the day before
				\item Analyze data and find low frequency earthquakes
				\item Update the catalog
			\end{itemize}
		\end{itemize}
	\end{frame}

	\begin{frame}
		\frametitle{First step: Python package}
		\scriptsize{
		\begin{columns}[c]
			\begin{column}{5cm}
				\begin{block}{catalog}
				\begin{itemize}
					\item	Specific Python scripts for downloading data and finding LFEs
					\item Directory with templates for two LFE families
				\end{itemize}
				\end{block}
				\begin{block}{utils}
				General Python scripts for stacking and cross-correlation
				\end{block}
				\begin{block}{data}
				\begin{itemize}
					\item File with list of LFE families
					\item File with list of seismic stations
					\item Directory with instrument response from seismic stations
				\end{itemize}
				\end{block}
			\end{column}
			\begin{column}{1cm}
				\centering
				\Huge\pointer
			\end{column}
			\begin{column}{5cm}
				\begin{block}{lfelib}
				\begin{itemize}
					\item Specific Python scripts for downloading data and finding LFEs
					\item \textsf{data} : Templates, instrument responses, list of LFE families and seismic stations
					\item \textsf{utils} : General Python scripts for stacking and cross-correlation
				\end{itemize}
				\end{block}
				\begin{block}{tests}
				\end{block}
				\begin{block}{.github/workflows}
				\end{block}
				\begin{itemize}
					\item environment.yml
					\item pyproject.toml
				\end{itemize}
			\end{column}
		\end{columns}
		}
%		\begin{itemize}
%			\item Reorganize the Python scripts to create a Python package
%			\item Each version available online
%			\item Input data and parameters easy available as well as the codes
%			\item Can be downloaded easily to be run on a daily basis
%		\end{itemize}
	\end{frame}

	\begin{frame}
		\frametitle{Second step: Command line}
		\begin{itemize}
			\item How to launch the scripts in a way that is both easy and fast?
			\item Create command lines to run Python functions
		\end{itemize}
	\end{frame}

	\begin{frame}
		\frametitle{Third step: GitHub workflow}
		\begin{itemize}
			\item Set up a workflow on GitHub
			\begin{itemize}
				\item Check last version of GitHub repository
				\item Get the appropriate version of Python
				\item Install the Python package
				\item Run the script
				\item Save the results
			\end{itemize}
		\end{itemize}		
	\end{frame}

	\begin{frame}
		\frametitle{Fourth step: Improving memory and computing time}
		\begin{itemize}
			\item Instead of storing instrument response, download them before analyzing the data
			\item Analysis of several low-frequency earthquake families:
			\begin{itemize}
				\item Loop on families: download data relevant to that family, analyze data, delete data
				\item Loop on seismic stations: Download all data, store them, loop on families, delete data

				$\rightarrow$ Make sure that each chunk of data is downloaded only once

			\end{itemize}
		\end{itemize}
	\end{frame}

	\begin{frame}
		\frametitle{Fifth step: Saving results}
		To be completed
	\end{frame}

	\begin{frame}
		\frametitle{Future improvements}
		\begin{itemize}
			\item Increase computing time by parallelization of the Python scripts
			\item GitHub CronJob offers a limited computing time $\rightarrow$ To analyze more data, we may use Amazon Lambda instead
		\end{itemize}
	\end{frame}

%What does not work:
%
%conda activate lfelib-dev
%poetry install
%poetry run pytest
%
%conda activate lfelib
%pip install --extra-index-url https://test.pypi.org/simple/ lfelib
%./myscript
%
%What does work:
%
%conda activate lfelib-dev
%./myscript
%
%One pull request remaining: What have you done? Can I merge?
%
%Modifying the CronJob
%
%Send output file somewhere
%
%I will do the parallelization later (no need for incubator presentation)
%
%I will look at Amazon Lambda later (when I have templates for more LFE families)

\end{document}
